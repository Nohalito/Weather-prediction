\documentclass[12pt,a4paper]{article}

% -------------------------
% Packages
% -------------------------
\usepackage[utf8]{inputenc}
\usepackage[T1]{fontenc}
\usepackage{lmodern}
\usepackage{setspace}
\usepackage{geometry}
\usepackage{amsmath, amssymb}
\usepackage{graphicx}
\usepackage{booktabs}
\usepackage[hidelinks]{hyperref}
\usepackage{float}


\geometry{margin=1in}
\onehalfspacing

% -------------------------
% Title Information
% -------------------------
\title{
\textbf{Impact of Weather on Energy Demand:\\
Cross-Section Prediction and Time-Series Forecasting}
}

\author{
Written and presented by\\
\vspace{0.3cm}
\textbf{R. AL AFIA, S. AUGUSTE, N. BOIMOND} \\
\textbf{D. Mamadou, P. OUEDRAOGO} \\
\vspace{0.5cm}
Under the supervision of\\
Professor Christophe MULLER\\
\vspace{0.5cm}
Predictive Methods Course
}

% -------------------------
% Document
% -------------------------
\begin{document}

\maketitle
\thispagestyle{empty}

\begin{figure}[h]
\centering
\includegraphics[width=0.5\textwidth]{assets/amse_logo_couleur_h600.png}
\label{fig:AMSE_logo}
\end{figure}

\newpage

\tableofcontents
\newpage

% ------------------------ Introduction ------------------------
\section{Introduction}
\subsection{General context}

Energy demand is a critical factor in modern infrastructure, influencing everything from 
economic stability to environmental sustainability. As global energy systems evolve, understanding 
the variables that drive consumption becomes increasingly important. Among these variables, weather 
stands out as a powerful and dynamic influence. Temperature fluctuations, precipitation, wind patterns, 
and seasonal changes directly impact heating, cooling, and overall electricity usage both at the 
household and industrial levels.

\subsection{Motivation and research questions}

This report explores the impact of weather on energy demand, employing two complementary analytical 
approaches: cross-section prediction and time-series forecasting. By examining how weather variables 
correlate with energy consumption across different regions and over time, we aim to answer key questions: 
How do specific weather conditions alter energy demand patterns? Can we accurately predict future 
demand based on weather forecasts? What are the implications for energy providers, policymakers, 
and consumers ?

The significance of this topic lies in its potential to enhance energy efficiency, optimize resource 
allocation, and mitigate the risks of supply shortages or surpluses. As climate change intensifies 
weather variability, the ability to forecast demand with precision becomes not only valuable but 
essential. Through this analysis, we seek to provide actionable insights for stakeholders in the 
energy sector, contributing to more resilient and sustainable energy systems.

% ------------------------ Data description ------------------------
\section{Data Description}

\subsection{Data Sources}

The data used in this study come from the Météo France\cite{Whatever 1} database, accessed through the official 
French open data platform data.gouv.fr. Météo France is the national meteorological service of 
France and provides The data used in this project come from two main open data sources: Météo France 
and Open Data Réseaux Énergies (ODRÉ)\cite{Whatever 2}.
Both sources are publicly available through official French open data platforms and provide 
reliable, high-quality information relevant to the study of energy demand and its relationship 
with weather conditions.

Meteorological data were obtained from the Météo France database, hosted by \href{https://www.data.gouv.fr}{the French 
official open data portal}.
This dataset contains daily observations collected by weather stations 
located across metropolitan France over the period from 2013 to 2023. The data include several 
weather indicators such as temperature, precipitation, and other atmospheric variables. In this 
study, particular attention is given to daily mean temperature, as it is a key determinant of 
heating and cooling needs and, consequently, energy demand.

In addition to meteorological data, energy-related data were sourced from the Open Data Réseaux 
Énergies (ODRÉ) platform. ODRÉ is a public initiative that provides open-access data related to 
energy production, consumption, infrastructure, markets, and territories in France. 

By combining meteorological data from Météo France with energy-related data from ODRÉ, this project 
relies on complementary sources that allow for a more comprehensive analysis of the interaction 
between weather conditions and energy demand. Both datasets are collected on a regular basis, 
cover several years, and are suitable for quantitative analysis, which makes them well adapted 
to the objectives of this study.

\subsection{Data description and preprocessing}

\subsubsection{Weather}

The dataset from météo France can be characterized as panel data, combining observations 
across time and across spatial units (weather stations and departments). Each observation 
corresponds to a specific station on a given day, which results in a large number of observations 
but also introduces heterogeneity in data availability across stations.

The raw dataset includes several meteorological indicators such as temperature, 
precipitation, wind speed, global radiation, sunshine duration, and humidity. Due 
to differences in station activity and reporting practices, some stations exhibit substantial 
missing values. To ensure data quality and temporal consistency, a station-level 
completion rate was computed for each variable. Only stations with a completion rate 
of at least 80\% were retained for further analysis, which limits the influence 
of inactive or unreliable stations.

For the purpose of this study, two different data structures were constructed 
depending on the modeling objective. For time-series analysis, the data were reshaped 
to obtain daily temperature series at the department level for the Provence–Alpes–Côte d’Azur 
(PACA) region. After filtering active stations, daily temperatures were aggregated by 
department using the mean across stations. In this setting, the data take the form 
of multivariate time-series data, where each department is observed repeatedly over 
time and each observation represents the daily average temperature of a given department 
on a specific date.

For cross-sectional prediction models, the data were organized differently. The 
full metropolitan area was retained in order to capture spatial variability across 
departments. Multiple meteorological variables were selected as explanatory features, while 
temperature was used as the target variable. In this case, each observation corresponds 
to a department-day combination, which allows the analysis of relationships between 
temperature and weather-related variables across space rather than over time.

After aggregation, a missing value analysis revealed only a limited number of remaining 
missing observations, mainly affecting global radiation and sunshine duration. 
Given their small proportion, these values were imputed using the mean of the corresponding 
variables. The final meteorological dataset was then sorted chronologically and stored 
in a processed format.

Overall, the dataset combines temporal, spatial, and quantitative dimensions, which 
makes it particularly rich but also requires careful preprocessing. The distinction 
between time-series data and cross-sectional data allows the project to address complementary 
research questions, while ensuring that the data structure is well adapted to each 
modeling approach.


\begin{table}[ht]
\centering
\begin{tabular}{l l r}
\hline
Date & Department name & Temperature \\
\hline
2013-01-01 & Alpes-de-Haute-Provence & 0.515 \\
2013-01-01 & Alpes-Maritimes & 3.134 \\
\vdots & \vdots & \vdots \\
2013-01-02 & Alpes-de-Haute-Provence & 1.084 \\
2013-01-02 & Alpes-Maritimes & 4.848 \\
\vdots & \vdots & \vdots \\
\hline
\end{tabular}
\caption{Panel data : Daily temperatures by department of the P.A.C.A. region}
\end{table}

\subsubsection{Energy}

The energy data used in this study take the form of aggregated univariate time-series data. Each
observation represents total daily energy consumption in the Provence–Alpes–Côte d’Azur
(PACA) region on a given date. The data are ordered over time and recorded at a daily frequency,
making them well suited for temporal analysis and forecasting.
In their raw form, the energy data include daily observations of electricity and gas consumption
reported in megawatts (MW) at the regional level. Together, these variables provide a
comprehensive measure of energy usage across different energy carriers. As an initial
preprocessing step, the data were cleaned and the date variable was converted into a datetime
format. The analysis was then restricted to the PACA region to ensure geographical consistency
with the meteorological data.

Total daily energy consumption was constructed by summing gross electricity consumption and
gross gas consumption, resulting in a single indicator of overall energy demand. The data were
subsequently aggregated at the daily level and filtered to match the study period starting in 2013.

Finally, the processed energy dataset was merged with the meteorological dataset using the date
variable, producing a temporally aligned and consistent dataset suitable for analyzing the
relationship between weather conditions and energy demand.

\begin{table}[ht]
\centering
\begin{tabular}{l c c c r}
\hline
$Y$: Energy\_cons\_MW & $\beta_1$: Temperature & $\cdots$ & $\beta_6$: Humidity \\
\hline
185715.0 & 4.795 & $\cdots$ & 4.882 \\
267200.0 & 4.476 & $\cdots$ & 0.038 \\
281535.0 & 4.670 & $\cdots$ & 0.0201 \\
$\vdots$ & $\vdots$ & $\vdots$ & $\vdots$ \\
\hline
\end{tabular}
\caption{Prediction datasets shape}
\end{table}

% ------------------------ EDA ------------------------
\section{Explanatory data analysis}

XXXXXXXXXXXXXXXXXXXXXXX

% ------------------------ Feature selection ------------------------
\section{Variable selection and empirical strategy}

\subsection{Energy : cross-section analysis}

For the cross section part (energy) of this project, the dependent variable in this study is total
energy consumption, denoted $Y_i$ , measured in megawatts (MW). It corresponds to the
variable $total\_energy\_consumption\_MW`$ in the final dataset. It represents the sum of gross
electricity consumption and gross gas consumption, aggregated at a daily frequency for the PACA
region. Energy consumption is a key economic indicator, as it reflects households’ and firms’
demand for heating, cooling, lighting, and other energy-intensive activities. From a statistical
perspective, the variable exhibits substantial temporal variability and pronounced seasonal
patterns, making it well suited for econometric analysis and prediction.

The explanatory variables are derived from daily meteorological observations provided by Météo
France. They include average daily temperature (TM, in °C), average humidity (UM, in \%),
precipitation (RR, in mm), wind speed (FFM, in m/s), global solar radiation (GLOT, in Wh/m²), and
sunshine duration (INST, in hours)\cite{Whatever 3}. These variables are expected to influence energy consumption
through several channels: temperature directly affects heating and cooling demand, humidity
modifies thermal comfort, while solar radiation and sunshine duration impact lighting needs and
energy production conditions. Wind and precipitation may also indirectly affect energy demand by
influencing perceived temperature and usage behavior.
To account for potential non-linear relationships, the empirical specification includes a quadratic
term in temperature and an interaction term between temperature and humidity, allowing the effect
of temperature on energy consumption to vary with humidity levels. All continuous variables
involved in non-linear terms are centered to improve coefficient interpretability and reduce
multicollinearity. The empirical strategy consists in modeling the conditional mean of energy
consumption as a function of meteorological variables and comparing several predictive models
based on their out-of-sample performance, using RMSE and MAE as evaluation criteria.

\subsection{Time-series analysis}

For the times series part (weather) of this project, the variable of interest is the average daily
temperature, denoted $TM_t$, measured in degrees Celsius (°C) and observed at a regular daily
frequency. 
The temperature series is obtained by averaging department-level observations across
France, resulting in a single national temperature time series. 
The objective is to predict future
temperature values $TM_{t+h}$ using only past information contained in the series itself, in line with
standard univariate time-series forecasting frameworks.

From a statistical perspective, this variable exhibits strong seasonal patterns and serial
dependence, which are characteristic features of meteorological time series. 
From an applied standpoint, temperature forecasting is a central task in meteorology and is highly relevant for
downstream applications such as energy demand forecasting, agriculture, and climate-related
decision-making.

About the explanatory variables of weather, they consist exclusively of past realizations of the
dependent variable itself. Specifically, lagged values $\{TM_{t-1}, TM_{t-2}, \cdots\}$ are used to capture
short-term temporal dependence, reflecting the persistence of weather conditions over
consecutive days. 
Moving-average components are introduced to model the dependence structure
of forecast errors, allowing the model to account for shocks that affect temperature temporarily.

Given the strong annual seasonality of temperature data, seasonal autoregressive and
moving-average terms are also included, associated with a yearly periodicity. These components
capture recurring patterns linked to the calendar, such as warmer summers and colder winters.
Preliminary graphical analysis, including the raw series and moving averages, suggests that
seasonality is stable over time and that no dominant deterministic long-term trend is present.

The empirical strategy consists of identifying the appropriate temporal structure of the series
before specifying a forecasting model. This includes assessing stationarity, serial correlation, and
seasonal patterns. Based on these properties, alternative specifications such as ARMA, ARIMA,
or SARIMA models are considered. Model performance is evaluated out of sample using standard
forecasting accuracy criteria, including the Root Mean Squared Error (RMSE) and the Mean
Absolute Error (MAE). This approach ensures that the selected model provides a statistically
sound and empirically reliable representation of temperature dynamics over time.

% ------------------------ Econometric Modeling ------------------------
\section{Econometric Modeling}

\subsection{Energy consumption}

\subsubsection{Model specification}

Several predictive models are considered to explain total energy consumption using
meteorological variables. The baseline specification relies on a linear regression model estimated
by Ordinary Least Squares (OLS), which serves as a reference framework for both interpretation
and comparison with more flexible approaches. \par
\textit{Baseline Model: Linear Regression (OLS)} \\
The reference model assumes a linear relationship between energy consumption and weather
conditions. The econometric specification is given by : \par
$Y_i = \beta_0 + \beta_1 TM_i + \beta_2 TM_i^2 + \beta_3 UM_i + \beta_4 RR_i + \beta_5 FFM_i + \beta_6 GLOT_i + \beta_7 INST_i + \beta_8 (TM_i \times UM_i) + u_i $ \hfill (1) \\
Where $Y_i$ denotes total daily energy consumption measured in megawatts (MW), and $u_i$ is an error
term capturing unobserved factors. The model is estimated under the standard exogeneity
assumption $\textbf{E}(u_i \mid X_i) = 0$, implying that the linear specification aims to approximate the conditional
mean of energy consumption given the meteorological variables.

The inclusion of a quadratic temperature term allows for a non-linear relationship between
temperature and energy demand, reflecting increased consumption during both cold and hot
extremes due to heating and cooling needs. The interaction term between temperature and
humidity captures the idea that thermal discomfort—and thus energy demand—may increase
more strongly when high temperatures coincide with high humidity levels. Variables related to
precipitation, wind speed, solar radiation, and sunshine duration account for additional channels
through which weather conditions can affect energy usage, such as heat losses, lighting needs,
and solar gains.\par
\textit{Penalized Regression Models: Ridge and Lasso} \\
Meteorological variables are often highly correlated, particularly temperature, solar radiation, and
sunshine duration. To address potential multicollinearity and improve out-of-sample predictive
performance, penalized regression models are also considered. Ridge regression introduces an 𝐿2
penalty that shrinks coefficients toward zero while retaining all regressors, thereby stabilizing
estimation in the presence of strong correlations. Lasso regression relies on an penalty, which𝐿1
can set some coefficients exactly to zero and thus performs variable selection in addition to
regularization.

All models are evaluated based on their predictive performance using out-of-sample criteria. The
primary loss functions considered are the Root Mean Squared Error (RMSE) and the Mean
Absolute Error (MAE), in accordance with the project guidelines. Comparing OLS with Ridge and
Lasso allows assessing the trade-off between interpretability and predictive accuracy in explaining
energy consumption from meteorological conditions.

\subsubsection{Estimation method}

In the cross-sectional framework devoted to energy consumption forecasting, several estimation
methods are considered in order to compare their predictive performance and to account for the
statistical properties of the meteorological explanatory variables.
The benchmark estimation method is Ordinary Least Squares (OLS). Under the standard
assumption of conditional exogeneity, $\textbf{E}(u_i \mid X_i) = 0$, 
the OLS estimator is unbiased and consistent. It provides a natural reference model, as it allows a
direct economic interpretation of the estimated coefficients and makes it possible to assess the
relevance of nonlinear effects, such as the quadratic term in temperature and the interaction
between temperature and humidity. In particular, OLS identifies average marginal effects of
meteorological variables on energy consumption.
However, several explanatory variables are potentially highly correlated, notably temperature, its
squared term, sunshine duration, and global radiation. This multicollinearity may inflate the
variance of OLS estimators and weaken out-of-sample predictive performance. To address this
issue, penalized regression methods are also implemented.

\textbf{Ridge regression} introduces an $L_2$ penalty on the coefficients and is defined as : \par

\hfill $\hat{\beta}^R = \textit{argmin}_{\beta} \sum_{i}^{}(Y_i - X_i\beta)^2 + \lambda\sum_{j \geq 1}^{}\beta_j^2$. \hfill (2) \\
This penalization reduces the variance of the estimators in the presence of multicollinearity, at the
cost of a controlled bias, and improves the stability of the estimated coefficients.

\textbf{Lasso regression} relies on an $L_1$ penalty and is defined as:\par

\hfill $\hat{\beta}^L = \textit{argmin}_{\beta} \sum_{i}^{}(Y_i - X_i\beta)^2 + \lambda\sum_{j \geq 1}^{}\mid \beta_j \mid$. \hfill (3) \\
Unlike Ridge regression, Lasso performs automatic variable selection by allowing some
coefficients to be exactly zero. This feature is particularly useful when not all meteorological
variables are equally relevant for predicting energy consumption.

Finally, model selection is based on out-of-sample predictive performance, in accordance with the
project guidelines. The OLS, Ridge, and Lasso models are compared using standard loss criteria
such as the Root Mean Squared Error (RMSE) and the Mean Absolute Error (MAE). The final
choice relies on a bias–variance trade-off and on the ability of the model to generalize beyond the
estimation sample.

\subsection{Weather}

\subsubsection{Stationarity assessment}

Before estimating a time-series model, it is necessary to assess the stationarity properties of the
temperature series. Time-series models such as ARMA, ARIMA, and SARIMA rely on the
assumption of stationarity, at least after possible transformations. The identification of unit roots
and seasonal patterns at this stage therefore directly guides the choice of the appropriate
modeling framework.
A preliminary visual inspection of the series suggests the presence of a pronounced and stable
annual seasonality. To formally test for stationarity, two complementary unit root tests are
employed: the Augmented Dickey-Fuller (ADF) test and the KPSS test. These tests are used
jointly because they rely on opposite null hypotheses, which strengthens the robustness of the
inference.
For the ADF test, the null hypothesis is that the temperature series contains a unit root and is
therefore non-stationary : \par
\hfill $H_0^{ADF}$ : the series is non-stationary. \hfill (H) \\
The alternative hypothesis is : \par
\hfill $H_1^{ADF}$ : the series is stationary. \hfill (H) \\
The Augmented Dickey-Fuller (ADF) test yields a test statistic of −4.29 with a p-value of 0.00047,
which is far below standard significance levels. Consequently, the null hypothesis of a unit root is
rejected, indicating that the temperature series is stationary in level.

For the KPSS test, the hypotheses are reversed. The null hypothesis assumes stationarity of the
series :

\hfill $H_0^{KPSS}$ : the series is stationary. \hfill (H) \\
The alternative hypothesis is : \par
\hfill $H_1^{KPSS}$ : the series is non-stationary. \hfill (H) \\
The KPSS test produces a test statistic of 0.077 with a p-value of 0.10. As a result, the null
hypothesis of stationarity cannot be rejected. The warning message indicates that the test statistic
is very small and that the true p-value is even larger than the reported value, which further
supports stationarity.

Taken together, the ADF and KPSS tests provide consistent evidence that the temperature series
is stationary, despite the presence of a stable annual seasonal pattern. Therefore, no additional
differencing is required before estimating time-series models, and seasonality can be directly
modeled within a SARIMA framework.

\begin{figure}[h]
\centering
\includegraphics[width=1.1\textwidth]{../results/figures/Diallo_SARIMA_illustration.png}
\caption{\textbf{SARIMA} illustration}
\label{fig:SARIMA}
\end{figure}

The figure illustrates the daily temperature series used for estimating the SARIMA model, along
with a 12-month moving average. The series is considered in levels, in line with the stationarity
test results. A pronounced and stable annual seasonality clearly emerges, while no strong
long-term deterministic trend is observed. This visual evidence supports the inclusion of seasonal
components in the SARIMA specification to capture the yearly temperature cycle.

\subsubsection{Model specification}

We consider a time-series forecasting framework for the monthly average temperature, 
denoted $TM_t$. As a baseline, a seasonal naïve forecasting model is used, defined by : \par
\hfill $\widehat{TM}_t = TM_{t-12}$ \hfill (4) \\
This approach predicts the temperature of a given month using the observed value from the same
month of the previous year. It provides a minimal benchmark against which the performance gains
of more sophisticated models can be evaluated. \par
\textit{SARIMA model} \\
To jointly capture short-term temporal dependence and the pronounced annual seasonality
observed in the data, a SARIMA model is considered. The specification is : \par
\hfill $\textbf{SARIMA}(1, 0, 1) \times (1, 1, 1)_{12}$ \hfill (5) \\
The non-seasonal components (1) model short-run dynamics, while the seasonal components (1) 
capture dependencies between observations separated by 12 months. The seasonal order $𝑠 = 12$
is imposed by the monthly frequency of the data.

Stationarity tests (ADF and KPSS) indicate that the series is stationary in levels, which justifies the
absence of non-seasonal differencing $(d = 0)$. However, given the strong and stable annual
seasonality, a first-order seasonal differencing $(D = 1)$ is introduced to stabilize the seasonal
pattern.

Overall, the inclusion of first-order autoregressive and moving-average components at both the
non-seasonal and seasonal levels allows the model to capture temporal dependence while
maintaining a parsimonious specification. The relevance of the SARIMA model is assessed by
comparing its out-of-sample forecasting performance to that of the seasonal naïve benchmark
using RMSE and MAE criteria.

\subsubsection{Estimation methods}

In the time-series framework, the SARIMA model is estimated using \textbf{Maximum Likelihood Estimation
(MLE)}. This method consists in selecting the model parameters that maximize the conditional likelihood of
the observed series, given the information available up to time $t - 1$.
Let $F_{t-1}$ denote the information set generated by past observations, and let $\epsilon_t$ be the innovation term. The
central assumption of the model is :\par
\hfill $\textbf{E}(\epsilon_t \mid F_{t-1}) = 0$ \hfill (6) \\
which implies that forecast errors are not predictable using past information.
Under the standard SARIMA assumptions, the innovations $\epsilon_t$ are uncorrelated, have zero mean and constant
variance, and are generally assumed to follow a Gaussian distribution for inference purposes. Under these
conditions, Maximum Likelihood Estimation provides efficient estimators of the model parameters.

MLE estimation also allows the direct derivation of \textbf{point forecasts} for the temperature series $TM_t$, as well
as prediction intervals, which quantify the uncertainty associated with the forecasts. In this project, the
model is estimated on a training sample, and its predictive performance is evaluated out-of-sample and
compared to that of the seasonal naïve benchmark using RMSE and MAE criteria. \\
 \\
\textit{Why Maximum Likelihood Estimation ?} \par
In time-series forecasting, the objective is not only to explain the variable of interest but primarily
to produce \textbf{optimal forecasts conditional on past information}. SARIMA models belong to the class
of \textbf{stochastic dynamic models}, in which the observed variable depends on its past values,
unobserved random innovations, and, when relevant, seasonal components.

In this context, Maximum Likelihood Estimation is the natural estimation method, as it relies
directly on the \textbf{conditional distribution of the series given past information}. The likelihood function
measures the probability of observing the realized trajectory of the series conditional on the model
parameters. Maximizing this likelihood therefore amounts to choosing the parameters that make
the observed data the most plausible, given the temporal dynamics imposed by the model.

This estimation method presents several key advantages in the context of this project. First, it
yields efficient estimators when the model is correctly specified. Second, it is well suited to
dynamic models featuring temporal dependence and seasonality. Finally, it allows the direct
construction of point forecasts and prediction intervals, which are essential for assessing forecast
uncertainty.

Overall, the use of Maximum Likelihood Estimation is fully consistent with the forecasting objective
of weather prediction and with the probabilistic structure of the SARIMA model.

% ------------------------ Estimation Results
\section{Estimation Results}

\subsection{Energy estimation output}

XXXXXXXXXXXXXXXXXXXXXXX

\subsection{Weather estimation output}

XXXXXXXXXXXXXXXXXXXXXXX

% ------------------------ Models Comparison ------------------------
\section{Comparison of Models and Choice of the Forecasting Method}

\subsection{Energy consumption prediction (Cross-Section)}

Several models were estimated to predict total energy consumption using meteorological
variables. The approaches considered include a linear regression estimated by Ordinary Least
Squares (OLS), as well as two penalized regression models, Ridge and Lasso.

Model selection is based exclusively on out-of-sample predictive performance, assessed using
forecasting loss functions adapted to prediction problems, namely the Root Mean Squared Error
(RMSE) and the Mean Absolute Error (MAE), in accordance with the project guidelines.

The results indicate that the predictive performances of the three models are overall very close.
The OLS model provides competitive forecasts, while the Ridge and Lasso models do not lead to
a substantial improvement in terms of RMSE or MAE on the test sample.

In this context, the OLS model is retained as the reference model, as it offers a good compromise
between predictive accuracy, simplicity, and economic interpretability. The penalized models
nevertheless remain informative: Ridge regression helps stabilize coefficient estimates in the
presence of multicollinearity, while Lasso highlights that some meteorological variables have a
limited marginal contribution to prediction.

Consistent with the objective of the project, this choice is driven by predictive performance rather
than by strict causal interpretation of the estimated coefficients.

\subsection{Temperature forecasting (Time series)}

The second forecasting task focuses on predicting the monthly average temperature based solely
on its past temporal dynamics. Two approaches are compared :
\begin{enumerate}
    \item A seasonal naïve model, used as a benchmark.
    \item A SARIMA model, designed to capture both short-term temporal dependence and annual
seasonality observed in the data.
\end{enumerate}

The seasonal naïve model assumes that the temperature in a given month is equal to the
temperature observed in the same month of the previous year. It provides a simple but demanding
baseline against which more sophisticated models are evaluated.

The SARIMA model, estimated by maximum likelihood, exploits the information contained in both
short-run dynamics and the seasonal structure of the temperature series. Its specification is
guided by graphical analysis and stationarity tests, in line with the theoretical framework
developed in the course.

The comparison of out-of-sample forecasting performance shows that the SARIMA model
improves prediction accuracy relative to the naïve benchmark, particularly in terms of RMSE and
MAE. This result indicates that temperature dynamics cannot be reduced to a simple annual
repetition and that intra-annual temporal dependence contains valuable predictive information.

As a result, the SARIMA model is selected as the preferred forecasting method, as it provides a
better balance between predictive accuracy and a rigorous representation of the underlying
temporal structure of the data.

\subsection{Methodological remark}

In both forecasting exercises, the final model selection relies exclusively on out-of-sample predictive
performance criteria, rather than on in-sample goodness-of-fit measures or the individual statistical
significance of estimated parameters.

This methodological choice is fully consistent with the project instructions and with the spirit of the course,
which emphasize a rigorous evaluation of forecasting performance over causal interpretation of model
coefficients.

% ------------------------ Final forecast ------------------------
\section{Final forecasts and comparative analysis}

XXXXXXXXXXXXXXXXXXXXXXX

% ------------------------ Metrics & limits found------------------------
\section{Robustness analysis and model limitations}

XXXXXXXXXXXXXXXXXXXXXXX

% ------------------------ Conclusion ------------------------
\section{Final conclusion of the project}

XXXXXXXXXXXXXXXXXXXXXXX

% ------------------------ References ------------------------
\begin{thebibliography}{9}

\bibitem{Whatever 1}
Météo France. \href{https://www.data.gouv.fr/datasets/donnees-climatologiques-de-base-quotidiennes}
{\textit{Données climatologiques de base - quotidiennes}} Licence Ouverte / Open Licence version 2.0.

\bibitem{Whatever 2}
ODRE. \href{https://www.data.gouv.fr/datasets/consommation-quotidienne-brute-regionale}
{\textit{Consommation quotidienne brute régionale}} Licence Ouverte / Open Licence version 2.0.

\bibitem{Whatever 3}
\begin{itemize}
    \item \textbf{RR}: Quantité de précipitation tombée en 24 heures, mesurée en millimètres et 1/10.
    \item \textbf{TM}: Moyenne quotidienne des températures horaires, exprimée en degrés Celsius avec une précision de 1/10.
    \item \textbf{FFM}: Moyenne quotidienne de la force du vent, mesurée en mètres par seconde (m/s) et moyenne sur 10 minutes, à 10 m de hauteur.
    \item \textbf{INST}: Durée d'insolation quotidienne mesurée en minutes.
    \item \textbf{GLOT}: Rayonnement global quotidien, exprimé en joules par centimètre carré (J/cm$^2$).
    \item \textbf{UM}: Moyenne quotidienne des humidités relatives horaires, exprimée en pourcentage (\%).
\end{itemize}

\end{thebibliography}

\end{document}
%\bibitem{Whatever}
%SDES. \href{https://www.statistiques.developpement-durable.gouv.fr/edition-numerique/chiffres-cles-energie/fr/6-bilan-energetique-de-la-france}
%{\textit{Bilan énergétique de la France | Chiffres clés de l'énergie - Édition 2025.}}