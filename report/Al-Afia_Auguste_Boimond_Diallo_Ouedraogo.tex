\documentclass[12pt,a4paper]{article}

% -------------------------
% Packages
% -------------------------
\usepackage[utf8]{inputenc}
\usepackage[T1]{fontenc}
\usepackage{lmodern}
\usepackage{setspace}
\usepackage{geometry}
\usepackage{amsmath, amssymb}
\usepackage{graphicx}
\usepackage{booktabs}
\usepackage[hidelinks]{hyperref}
\usepackage{float}


\geometry{margin=1in}
\onehalfspacing

% -------------------------
% Title Information
% -------------------------
\title{
\textbf{Impact of Weather on Energy Demand:\\
Cross-Section Prediction and Time-Series Forecasting}
}

\author{
Written and presented by\\
\vspace{0.3cm}
\textbf{R. AL AFIA, S. AUGUSTE, N. BOIMOND} \\
\textbf{D. Mamadou, P. OUEDRAOGO} \\
\vspace{0.5cm}
Under the supervision of\\
Professor Christophe MULLER\\
\vspace{0.5cm}
Predictive Methods Course
}

% -------------------------
% Document
% -------------------------
\begin{document}

\maketitle
\thispagestyle{empty}

\begin{figure}[h]
\centering
\includegraphics[width=0.5\textwidth]{assets/amse_logo_couleur_h600.png}
\label{fig:AMSE_logo}
\end{figure}

\newpage

\tableofcontents
\newpage

% ------------------------ Introduction ------------------------
\section{Introduction}

\subsection{General context}

Energy demand plays a central role in modern economies, as it conditions economic activity,
infrastructure planning, and environmental sustainability. Accurate forecasting of energy consumption
is therefore a key concern for energy providers, policymakers, and system operators, particularly in a
context of increasing climate variability and energy transition. Among the many determinants of energy
demand, weather conditions stand out as a major and highly volatile factor. Temperature, humidity,
precipitation, wind, and solar exposure directly affect heating, cooling, lighting needs, and indirectly
influence industrial and household energy use.\par

At the same time, weather variables themselves—especially temperature—exhibit strong temporal
structures, characterized by persistence and pronounced seasonal patterns. Understanding and
forecasting these dynamics is essential, both as an object of interest in its own right and as an input for
energy demand forecasting. As a result, weather and energy forecasting naturally call for
complementary empirical approaches, combining cross-sectional prediction models and time series
methods.

\subsection{Motivation and research questions}

This project investigates the relationship between weather conditions and energy consumption using
two complementary forecasting frameworks. First, energy consumption is predicted in a cross-
sectional setting using meteorological variables as exogenous predictors. Second, temperature is
modeled and forecasted using univariate time series techniques, relying exclusively on its past
dynamics.

The analysis is guided by the following research questions :
\begin{enumerate}
    \item \textit{How well can energy consumption be predicted using meteorological information in a cross-
sectional framework ?}
    \item \textit{How effectively do time series models capture and forecast the seasonal dynamics of temperature ?}
    \item \textit{How do cross-sectional and time series approaches complement each other in forecasting
applications ?}
\end{enumerate}

Rather than aiming at causal inference, the focus of this project is strictly predictive. Models are
evaluated based on their out-of-sample forecasting performance, in line with modern forecasting
practice and course guidelines.

\subsection{Related literature}

A large body of empirical literature documents the strong link between weather conditions and energy
consumption. Early studies such as \textbf{Bessec and Fouquau (2008)} show that temperature–energy
relationships are inherently non-linear, reflecting asymmetric heating and cooling needs. More recent
contributions, including \textbf{Auffhammer et al. (2017)} and \textbf{Deschênes and Greenstone (2011)},
emphasize the growing sensitivity of energy demand to extreme temperatures in the context of climate
change.

From a methodological perspective, cross-sectional and panel regression models using weather
variables have been widely employed for short-term energy demand forecasting (see \textbf{Hong et al.,
2016}). These models are valued for their interpretability and their ability to incorporate rich
meteorological information, but they may suffer from multicollinearity and limited ability to capture
temporal dependence.

In parallel, time series models such as ARIMA and SARIMA have long been used to forecast
meteorological variables, particularly temperature, due to their strong seasonal structure (\textbf{Box et al.,
2015}). Seasonal naïve models provide a natural benchmark, but more flexible stochastic models often
yield superior predictive performance by exploiting short-term persistence in the data.

\subsection{Contribution and structure of the report}

The contribution of this project is twofold. First, it provides an empirical assessment of how far
meteorological variables alone can explain and predict daily energy consumption when non-linearities
and interactions are explicitly modeled. Second, it illustrates how univariate time series models
capture the seasonal and dynamic structure of temperature, and how this affects forecast accuracy
relative to simple benchmarks. The report is organized as follows :
\begin{itemize}
    \item \textbf{Part 1} : focuses on weather data and temperature forecasting using time series methods. It presents the
data, exploratory analysis, stationarity tests, model specification, estimation, and forecast evaluation.
    \item \textbf{Part 2} : examines energy consumption in a cross-sectional framework, including data description,
explanatory analysis, econometric modeling, estimation results, and model comparison.
    \item \textbf{Finally}, we will compare forecasting approaches, discuss robustness and limitations, and conclude
on the implications of the results for predictive modeling.
\end{itemize}

% ------------------------------------------------ Part 1 : Weather data ------------------------------------------------
\part{}
\section{Weather data \& forecasting}

% ------------------------ Data desc ------------------------
\subsection{Data description}
\subsubsection{Data sources}

The data used in this study come from the Météo France\cite{Meteo_france} database, accessed through the official
French open data platform data.gouv.fr. Météo France is the national meteorological service of France
and provides Meteorological data were obtained from the Météo France database, hosted on data.gouv.fr, the
French official open data portal. This dataset contains daily observations collected by weather stations
located across metropolitan France over the period from 2013 to 2023. The data include several
weather indicators such as temperature, precipitation, and other atmospheric variables. In this study,
particular attention is given to daily mean temperature, as it is a key determinant of heating and cooling
needs and, consequently, energy demand.

\subsubsection{Data description and pre-processing}

The dataset from météo France can be characterized as panel data, combining observations across
time and across spatial units (weather stations and departments). Each observation corresponds to a
specific station on a given day, which results in a large number of observations but also introduces
heterogeneity in data availability across stations.

The raw dataset includes several meteorological indicators such as temperature, precipitation, wind
speed, global radiation, sunshine duration, and humidity \cite{feat_desc}. Due to differences in station activity and
reporting practices, some stations exhibit substantial missing values. To ensure data quality and
temporal consistency, a station-level completion rate was computed for each variable. Only stations
with a completion rate of at least 80\% were retained for further analysis, which limits the influence of
inactive or unreliable stations.

For the purpose of this study, two different data structures were constructed depending on the
modeling objective. For time-series analysis, the data were reshaped to obtain daily temperature
series at the department level for the Provence–Alpes–Côte d’Azur (PACA) region. After filtering active
stations, daily temperatures were aggregated by department using the mean across stations. In this
setting, the data take the form of multivariate time-series data, where each department is observed
repeatedly over time and each observation represents the daily average temperature of a given
department on a specific date.

For cross-sectional prediction models, the data were organized differently. The full metropolitan area
was retained in order to capture spatial variability across departments. Multiple meteorological
variables were selected as explanatory features, while temperature was used as the target variable. In
this case, each observation corresponds to a department-day combination, which allows the
analysis of relationships between temperature and weather-related variables across space rather than
over time.

After aggregation, a missing value analysis revealed only a limited number of remaining missing
observations, mainly affecting global radiation and sunshine duration. Given their small proportion,
these values were imputed using the mean of the corresponding variables. The final meteorological
dataset was then sorted chronologically and stored in a processed format.

Overall, the dataset combines temporal, spatial, and quantitative dimensions, which makes it
particularly rich but also requires careful preprocessing. The distinction between time-series data and
cross-sectional data allows the project to address complementary research questions, while ensuring
that the data structure is well adapted to each modeling approach.

\begin{table}[ht]
\centering
\begin{tabular}{l l r}
\hline
Date & Department name & Temperature \\
\hline
2013-01-01 & Alpes-de-Haute-Provence & 0.515 \\
2013-01-01 & Alpes-Maritimes & 3.134 \\
\vdots & \vdots & \vdots \\
2013-01-02 & Alpes-de-Haute-Provence & 1.084 \\
2013-01-02 & Alpes-Maritimes & 4.848 \\
\vdots & \vdots & \vdots \\
\hline
\end{tabular}
\caption{Panel data : Daily temperatures by department of the P.A.C.A. region}
\end{table}

% ------------------------ EDA ------------------------
\subsection{Exploratory Data Analysis}

With the time series forecast exercise, we will try to check if there is 
something to predict (seasonality, trend or cyclicity) :\par

\begin{figure}[h]
\centering
\includegraphics[width=1\textwidth]{../results/figures/Noa_dept_serie.png}
\caption{Time series of the Bouche du Rhône average temperature}
\label{fig:serie}
\end{figure}

\begin{figure}[h]
\centering
\includegraphics[width=0.8\textwidth]{../results/figures/Noa_Montly_temp_dept.png}
\caption{Seasonal patern}
\label{fig:season}
\end{figure}

These figures (1 \& 2) reveal 2 things :
\begin{itemize}
    \item Identical shape and patern (Up and down)
    \item Fixed period of repetition (12 months)
\end{itemize}
This is exactly the definition of a seasonality ! Thus this will be our target for the 
forecast exercise. We can further add that there is a trend (although almost invisible to the naked eye) of 
rise in temperature, this could illustrate perfectly global warming.

% ------------------------ Variable selection and empirical strategy ------------------------
\subsection{Variable selection and empirical strategy}

For the times series part (weather) of this project, the variable of interest is the average daily
temperature, denoted $TM_t$, measured in degrees Celsius (°C) and observed at a regular daily
frequency. 
The temperature series is obtained by averaging department-level observations across
France, resulting in a single national temperature time series. 
The objective is to predict future
temperature values $TM_{t+h}$ using only past information contained in the series itself, in line with
standard univariate time-series forecasting frameworks.

From a statistical perspective, this variable exhibits strong seasonal patterns and serial
dependence, which are characteristic features of meteorological time series. 
From an applied standpoint, temperature forecasting is a central task in meteorology and is highly relevant for
downstream applications such as energy demand forecasting, agriculture, and climate-related
decision-making.

About the explanatory variables of weather, they consist exclusively of past realizations of the
dependent variable itself. Specifically, lagged values $\{TM_{t-1}, TM_{t-2}, \cdots\}$ are used to capture
short-term temporal dependence, reflecting the persistence of weather conditions over
consecutive days. 
Moving-average components are introduced to model the dependence structure
of forecast errors, allowing the model to account for shocks that affect temperature temporarily.

Given the strong annual seasonality of temperature data, seasonal autoregressive and
moving-average terms are also included, associated with a yearly periodicity. These components
capture recurring patterns linked to the calendar, such as warmer summers and colder winters.
Preliminary graphical analysis, including the raw series and moving averages, suggests that
seasonality is stable over time and that no dominant deterministic long-term trend is present.

The empirical strategy consists of identifying the appropriate temporal structure of the series
before specifying a forecasting model. This includes assessing stationarity, serial correlation, and
seasonal patterns. Based on these properties, alternative specifications such as ARMA, ARIMA,
or SARIMA models are considered. Model performance is evaluated out of sample using standard
forecasting accuracy criteria, including the Root Mean Squared Error (RMSE) and the Mean
Absolute Error (MAE). This approach ensures that the selected model provides a statistically
sound and empirically reliable representation of temperature dynamics over time.

% ------------------------ Econometric modeling ------------------------
\subsection{Econometric modeling}
\subsubsection{Stationnary assessement}

Before estimating a time-series model, it is necessary to assess the stationarity properties of the
temperature series. Time-series models such as ARMA, ARIMA, and SARIMA rely on the
assumption of stationarity, at least after possible transformations. The identification of unit roots
and seasonal patterns at this stage therefore directly guides the choice of the appropriate
modeling framework.

A preliminary visual inspection of the series suggests the presence of a pronounced and stable
annual seasonality. To formally test for stationarity, two complementary unit root tests are
employed: the Augmented Dickey-Fuller (ADF) test and the KPSS test. These tests are used
jointly because they rely on opposite null hypotheses, which strengthens the robustness of the
inference.

For the ADF test, the null hypothesis is that the temperature series contains a unit root and is
therefore non-stationary :
\begin{center}
    $H_0^{ADF}$ : the series is non-stationary.
\end{center}
The alternative hypothesis is :
\begin{center}
    $H_1^{ADF}$ : the series is stationary.
\end{center}
The Augmented Dickey-Fuller (ADF) test yields a test statistic of −4.29 with a p-value of 0.00047,
which is far below standard significance levels. Consequently, the null hypothesis of a unit root is
rejected, indicating that the temperature series is stationary in level.

For the KPSS test, the hypotheses are reversed. The null hypothesis assumes stationarity of the
series :
\begin{center}
    $H_0^{KPSS}$ : the series is stationary.
\end{center}
The alternative hypothesis is :
\begin{center}
    $H_1^{KPSS}$ : the series is non-stationary.
\end{center}
The KPSS test produces a test statistic of 0.077 with a p-value of 0.10. As a result, the null
hypothesis of stationarity cannot be rejected. The warning message indicates that the test statistic
is very small and that the true p-value is even larger than the reported value, which further
supports stationarity.

Taken together, the ADF and KPSS tests provide consistent evidence that the temperature series
is stationary, despite the presence of a stable annual seasonal pattern. Therefore, no additional
differencing is required before estimating time-series models, and seasonality can be directly
modeled within a SARIMA framework.

\begin{figure}[h]
\centering
\includegraphics[width=1.1\textwidth]{../results/figures/Diallo_SARIMA_illustration.png}
\caption{\textbf{SARIMA} illustration}
\label{fig:SARIMA}
\end{figure}

The figure illustrates the daily temperature series used for estimating the SARIMA model, along
with a 12-month moving average. The series is considered in levels, in line with the stationarity
test results. A pronounced and stable annual seasonality clearly emerges, while no strong
long-term deterministic trend is observed. This visual evidence supports the inclusion of seasonal
components in the SARIMA specification to capture the yearly temperature cycle.

\subsubsection{Model specification}

We consider a time-series forecasting framework for the monthly average temperature, 
denoted $TM_t$. As a baseline, a seasonal naïve forecasting model is used, defined by :
\begin{center}
    $\widehat{TM}_t = TM_{t-12}$
\end{center}
This approach predicts the temperature of a given month using the observed value from the same
month of the previous year. It provides a minimal benchmark against which the performance gains
of more sophisticated models can be evaluated. \par
\textit{SARIMA model} \\
To jointly capture short-term temporal dependence and the pronounced annual seasonality
observed in the data, a SARIMA model is considered. The specification is :
\begin{center}
    $\textbf{SARIMA}(1, 0, 1) \times (1, 1, 1)_{12}$
\end{center}
The non-seasonal components (1,0,1) model short-run dynamics, while the seasonal components (1,1,1) 
capture dependencies between observations separated by 12 months. The seasonal order $𝑠 = 12$
is imposed by the monthly frequency of the data.

Stationarity tests (ADF and KPSS) indicate that the series is stationary in levels, which justifies the
absence of non-seasonal differencing $(d = 0)$. However, given the strong and stable annual
seasonality, a first-order seasonal differencing $(D = 1)$ is introduced to stabilize the seasonal
pattern.

Overall, the inclusion of first-order autoregressive and moving-average components at both the
non-seasonal and seasonal levels allows the model to capture temporal dependence while
maintaining a parsimonious specification. The relevance of the SARIMA model is assessed by
comparing its out-of-sample forecasting performance to that of the seasonal naïve benchmark
using RMSE and MAE criteria.

\subsubsection{Estimation methods}

In the time-series framework, the SARIMA model is estimated using \textbf{Maximum Likelihood Estimation
(MLE)}. This method consists in selecting the model parameters that maximize the conditional likelihood of
the observed series, given the information available up to time $t - 1$.
Let $F_{t-1}$ denote the information set generated by past observations, and let $\epsilon_t$ be the innovation term. The
central assumption of the model is :\par
\hfill $\textbf{E}(\epsilon_t \mid F_{t-1}) = 0$ \hfill (6) \\
which implies that forecast errors are not predictable using past information.

Under the standard SARIMA assumptions, the innovations $\epsilon_t$ are uncorrelated, have zero mean and constant
variance, and are generally assumed to follow a Gaussian distribution for inference purposes. Under these
conditions, Maximum Likelihood Estimation provides efficient estimators of the model parameters.

MLE estimation also allows the direct derivation of \textbf{point forecasts} for the temperature series $TM_t$, as well
as prediction intervals, which quantify the uncertainty associated with the forecasts. In this project, the
model is estimated on a training sample, and its predictive performance is evaluated out-of-sample and
compared to that of the seasonal naïve benchmark using RMSE and MAE criteria. \\
 \\
\textit{Why Maximum Likelihood Estimation ?} \par
In time-series forecasting, the objective is not only to explain the variable of interest but primarily
to produce \textbf{optimal forecasts conditional on past information}. SARIMA models belong to the class
of \textbf{stochastic dynamic models}, in which the observed variable depends on its past values,
unobserved random innovations, and, when relevant, seasonal components.

In this context, Maximum Likelihood Estimation is the natural estimation method, as it relies
directly on the \textbf{conditional distribution of the series given past information}. The likelihood function
measures the probability of observing the realized trajectory of the series conditional on the model
parameters. Maximizing this likelihood therefore amounts to choosing the parameters that make
the observed data the most plausible, given the temporal dynamics imposed by the model.

This estimation method presents several key advantages in the context of this project. First, it
yields efficient estimators when the model is correctly specified. Second, it is well suited to
dynamic models featuring temporal dependence and seasonality. Finally, it allows the direct
construction of point forecasts and prediction intervals, which are essential for assessing forecast
uncertainty.

Overall, the use of Maximum Likelihood Estimation is fully consistent with the forecasting objective
of weather prediction and with the probabilistic structure of the SARIMA model.

% ------------------------ Estimation results ------------------------
\subsection{Estimation results}

The second part of the analysis focuses on modeling and forecasting average monthly temperature.
The variable of interest is the average temperature (TM), observed at a monthly frequency from January
2013 to September 2021. The objective is to generate forecasts using only the information contained in
the past realizations of the series, within a univariate time series framework.

Exploratory analysis reveals a strong and stable annual seasonality, with systematic temperature
peaks during summer months and troughs during winter months, and no clear evidence of a
deterministic long-term trend. These visual findings are confirmed by formal stationarity tests: the
augmented Dickey–Fuller test rejects the null hypothesis of a unit root, while the KPSS test does not
reject the null hypothesis of stationarity. Taken together, these results indicate that the series can be
modeled in levels, provided that seasonality is explicitly incorporated.

On this basis, a SARIMA model is adopted to capture both short-term temporal dependence and
annual seasonal patterns. The selected specification is a SARIMA(1,0,1) $\times$ (1,1,1)$_{12}$, reflecting
monthly data with a twelve-month seasonal cycle. The model is estimated by maximum likelihood.

The estimation results highlight the importance of seasonal dynamics. Seasonal autoregressive and
moving average parameters are statistically significant, confirming strong dependence between
observations separated by one year. The non-seasonal moving average term is also significant,
indicating that temperature shocks have a short-lived impact on the series. The residual variance is
stable and consistent with observed fluctuations, suggesting that the model captures a substantial
share of temperature variability.

Model diagnostics support the validity of the specification. Residual autocorrelation tests do not
indicate remaining serial dependence, and normality tests do not reject the Gaussian assumption. No
evidence of heteroskedasticity is detected, reinforcing the appropriateness of maximum likelihood
estimation.

Forecasting performance is evaluated by comparing the SARIMA model to a seasonal naïve
benchmark, which assumes that temperature in a given month equals that of the same month in the
previous year. The SARIMA model consistently outperforms the benchmark in terms of RMSE and MAE
on the test sample. This improvement demonstrates that temperature dynamics cannot be reduced to
simple annual repetition and that short-term information significantly enhances predictive accuracy.

As in the cross-sectional analysis, the interpretation of the estimated parameters remains strictly
predictive. The results show that temperature follows a well-structured temporal process dominated
by stable seasonality and short-term dependence, and that SARIMA models provide an effective
framework for medium-term temperature forecasting.

% ------------------------ Comparison of Models and Choice of the Forecasting Method ------------------------
\subsection{Comparison of Models and Choice of the Forecasting Method}

The second forecasting task focuses on predicting the \textbf{monthly average temperature} based solely
on its past temporal dynamics. Two approaches are compared :
\begin{enumerate}
    \item A seasonal naïve model, used as a benchmark.
    \item A SARIMA model, designed to capture both short-term temporal dependence and annual
seasonality observed in the data.
\end{enumerate}

The seasonal naïve model assumes that the temperature in a given month is equal to the
temperature observed in the same month of the previous year. It provides a simple but demanding
baseline against which more sophisticated models are evaluated.

The SARIMA model, estimated by maximum likelihood, exploits the information contained in both
short-run dynamics and the seasonal structure of the temperature series. Its specification is
guided by graphical analysis and stationarity tests, in line with the theoretical framework
developed in the course.

The comparison of out-of-sample forecasting performance shows that the SARIMA model
improves prediction accuracy relative to the naïve benchmark, particularly in terms of RMSE and
MAE. This result indicates that temperature dynamics cannot be reduced to a simple annual
repetition and that intra-annual temporal dependence contains valuable predictive information.

As a result, the SARIMA model is selected as the preferred forecasting method, as it provides a
better balance between predictive accuracy and a rigorous representation of the underlying
temporal structure of the data.\\

\textbf{Methodological Remark} \par

The final model selection relies exclusively on out-of-sample predictive performance criteria, rather than on
in-sample goodness-of-fit measures or the individual statistical significance of estimated parameters.

This methodological choice is fully consistent with the project instructions and with the spirit of the course,
which emphasize a rigorous evaluation of forecasting performance over causal interpretation of model
coefficients.

\subsection{Final forecasts and comparative analysis}
Final temperature forecasts are produced using the time series model retained at the end of the selection
phase. Forecasts are generated over an out-of-sample horizon and constitute the final predictive output of the
temporal approach.

The predicted temperature trajectory extends the dynamics observed over the estimation period in a coherent
manner. Forecasts reproduce the strong annual seasonality of the series and follow the general trajectory of
observed values, while smoothing short-term fluctuations. This behavior is characteristic of univariate time
series models, which prioritize stability of the predicted trajectory over exact reproduction of transitory
shocks.

\begin{figure}[h]
\centering
\includegraphics[width=1\textwidth]{../results/figures/Diallo_S2_Q12_selected_forecasts_plot.png}
\caption{Observed and predicted values}
\label{fig:Obs_Vs_Pred}
\end{figure}

Beyond point forecasts, uncertainty associated with the forecasts is explicitly taken into account through the
construction of prediction intervals.

\begin{figure}[h]
\centering
\includegraphics[width=0.9\textwidth]{../results/figures/Diallo_S2_SARIMA_IC_graph.png}
\caption{Temperature forecast with 95\% prediction intervals}
\label{fig:IC95_forecast}
\end{figure}

The prediction intervals show that most out-of-sample observations fall within the confidence bands, while
their progressive widening as the forecast horizon increases reflects the natural accumulation of uncertainty
in a time series framework. \par

\textbf{Comparative Analysis} \par

The temperature forecasting exercise highlights the effectiveness of a univariate time-series approach
in capturing structured temporal dynamics. By relying exclusively on past realizations of the series, the
SARIMA model exploits strong seasonality and short-term dependence to generate forecasts that are
temporally coherent and stable. Compared to the seasonal naïve benchmark, this approach delivers
improved predictive accuracy, particularly at medium horizons. Nevertheless, the reliance on
historical patterns implies a smoothing of short-term fluctuations, limiting the model’s ability to
anticipate sudden or irregular temperature shocks. As a result, the time-series approach prioritizes
structural regularity over responsiveness to unexpected variations

\subsection{Robustness analysis and model limitations}

\textbf{Robustness of the results} \par
The robustness of temperature forecasts is evaluated by comparing alternative time-series
specifications and by examining the behavior of out-of-sample forecast errors. Models that explicitly
incorporate annual seasonality consistently generate stable and coherent predictive trajectories, with
no evidence of systematic bias. The SARIMA specification performs robustly relative to simpler
benchmarks, indicating that the main temporal dynamics of the temperature series are adequately
captured.\\
\\
\textbf{Model and data limitations} \par
Nevertheless, several limitations apply. Temperature forecasting relies on univariate time-series
models, which assume that all relevant predictive information is contained in past realizations of the
series. As a result, unexpected shocks or structural changes cannot be anticipated. Moreover, forecast
uncertainty increases with the prediction horizon, as reflected by widening prediction intervals, which
limits precision at medium and long horizons.\\
\\
\textbf{Implications for interpretation}\par
Temperature forecasts should therefore be interpreted as probabilistic projections of expected
seasonal patterns rather than precise point predictions. They are particularly informative for identifying
medium-term trends and seasonal behavior, while short-term irregular variations remain inherently
difficult to predict.

% ------------------------------------------------ Part 2 : Energy data & Prediction ------------------------------------------------
\part{}
\section{Energy data \& Prediction}

% ------------------------ Data desc ------------------------
\subsection{Data description}
\subsubsection{Data sources}

The data used in this part of the project come from Open Data Réseaux Énergies (ODRÉ) \cite{ODRE}. This source
is publicly available through official French open data platforms and provide reliable, high-quality
information relevant to the study of energy demand and its relationship with weather conditions.

Energy-related data were sourced from the Open Data Réseaux Énergies (ODRÉ) platform. ODRÉ is a
public initiative that provides open-access data related to energy production, consumption,
infrastructure, markets, and territories in France.

\subsubsection{Data description and pre-processing}

The energy data used in this study take the form of \textbf{aggregated univariate time-series data}. Each
observation represents total daily energy consumption in the Provence–Alpes–Côte d’Azur (PACA)
region on a given date. The data are ordered over time and recorded at a daily frequency, making them
well suited for temporal analysis and forecasting.

In their raw form, the energy data include daily observations of electricity and gas consumption
reported in megawatts (MW) at the regional level. Together, these variables provide a comprehensive
measure of energy usage across different energy carriers. As an initial preprocessing step, the data
were cleaned and the date variable was converted into a datetime format. The analysis was then
restricted to the PACA region to ensure geographical consistency with the meteorological data.

Total daily energy consumption was constructed by summing gross electricity consumption and gross
gas consumption, resulting in a single indicator of overall energy demand. The data were subsequently
aggregated at the daily level and filtered to match the study period starting in 2013. Finally, the
processed energy dataset was merged with the meteorological dataset using the date variable,
producing a temporally aligned and consistent dataset suitable for analyzing the relationship between
weather conditions and energy demand.

\begin{table}[ht]
\centering
\begin{tabular}{l c c c r}
\hline
$Y$: Energy\_cons\_MW & $\beta_1$: Temperature & $\cdots$ & $\beta_6$: Humidity \\
\hline
185715.0 & 4.795 & $\cdots$ & 4.882 \\
267200.0 & 4.476 & $\cdots$ & 0.038 \\
281535.0 & 4.670 & $\cdots$ & 0.0201 \\
$\vdots$ & $\vdots$ & $\vdots$ & $\vdots$ \\
\hline
\end{tabular}
\caption{Prediction datasets shape}
\end{table}

% ------------------------ EDA ------------------------
\subsection{Exploratory Data Analysis}

For the prediction part, we can observe the following table : \par

\begin{figure}[h]
\centering
\includegraphics[width=1\textwidth]{../results/figures/Noa_scatterplot_target.png}
\caption{Features scatterplot with the target}
\label{fig:scatterplot}
\end{figure}

As we had to restrict ourselve from using a lot of different features (due to extensively high
missing value rate in the weather dataset), we only retained 6 defaults $\beta$.
Here, a simple glance is enough to draw a few conclusion :
\begin{itemize}
    \item Some features are cleary uncorrelated to the target (a more in-depth analysis \cite{corr_matrix}) 
    \item We can expect to find non-linearity in the relation (J-shaped curve with $\beta_{TM}$)
\end{itemize}
Meaning, that coming up with prediction with this few information, might be challenging
in both a methodological and theoretical point of view (lack of meaning).

% ------------------------ Variable selection and empirical strategy ------------------------
\subsection{Variable selection and empirical strategy}

For the cross section part (energy) of this project, the dependent variable in this study is total
energy consumption, denoted $Y_i$ , measured in megawatts (MW). It corresponds to the
variable $total\_energy\_consumption\_MW$ in the final dataset. It represents the sum of gross
electricity consumption and gross gas consumption, aggregated at a daily frequency for the PACA
region. Energy consumption is a key economic indicator, as it reflects households’ and firms’
demand for heating, cooling, lighting, and other energy-intensive activities. From a statistical
perspective, the variable exhibits substantial temporal variability and pronounced seasonal
patterns, making it well suited for econometric analysis and prediction. \par

The explanatory variables are derived from daily meteorological observations provided by Météo
France. They include average daily temperature (TM, in °C), average humidity (UM, in \%),
precipitation (RR, in mm), wind speed (FFM, in m/s), global solar radiation (GLOT, in Wh/m²), and
sunshine duration (INST, in hours)\cite{feat_desc}. These variables are expected to influence energy consumption
through several channels: temperature directly affects heating and cooling demand, humidity
modifies thermal comfort, while solar radiation and sunshine duration impact lighting needs and
energy production conditions. Wind and precipitation may also indirectly affect energy demand by
influencing perceived temperature and usage behavior.\par

To account for potential non-linear relationships, the empirical specification includes a quadratic
term in temperature and an interaction term between temperature and humidity, allowing the effect
of temperature on energy consumption to vary with humidity levels. All continuous variables
involved in non-linear terms are centered to improve coefficient interpretability and reduce
multicollinearity. The empirical strategy consists in modeling the conditional mean of energy
consumption as a function of meteorological variables and comparing several predictive models
based on their out-of-sample performance, using RMSE and MAE as evaluation criteria.\par

% ------------------------ Econometric modeling ------------------------
\subsection{Econometric modeling}
\subsubsection{Model specification}

Several predictive models are considered to explain total energy consumption using
meteorological variables. The baseline specification relies on a linear regression model estimated
by Ordinary Least Squares (OLS), which serves as a reference framework for both interpretation
and comparison with more flexible approaches. \\
 \\
\textit{Baseline Model: Linear Regression (OLS)} \par
The reference model assumes a linear relationship between energy consumption and weather
conditions. The econometric specification is given by :
\begin{center}
    $Y_i = \beta_0 + \beta_1 TM_i + \beta_2 TM_i^2 + \beta_3 UM_i + \beta_4 RR_i + \beta_5 FFM_i + \beta_6 GLOT_i + \beta_7 INST_i + \beta_8 (TM_i \times UM_i) + u_i $
\end{center}
Where $Y_i$ denotes total daily energy consumption measured in megawatts (MW), and $u_i$ is an error
term capturing unobserved factors. The model is estimated under the standard exogeneity
assumption $\textbf{E}(u_i \mid X_i) = 0$, implying that the linear specification aims to approximate the conditional
mean of energy consumption given the meteorological variables. \\

The inclusion of a quadratic temperature term allows for a non-linear relationship between
temperature and energy demand, reflecting increased consumption during both cold and hot
extremes due to heating and cooling needs. The interaction term between temperature and
humidity captures the idea that thermal discomfort—and thus energy demand—may increase
more strongly when high temperatures coincide with high humidity levels. Variables related to
precipitation, wind speed, solar radiation, and sunshine duration account for additional channels
through which weather conditions can affect energy usage, such as heat losses, lighting needs,
and solar gains.\\
 \\
\textit{Penalized Regression Models: Ridge and Lasso} \par
Meteorological variables are often highly correlated, particularly temperature, solar radiation, and
sunshine duration. To address potential multicollinearity and improve out-of-sample predictive
performance, penalized regression models are also considered. Ridge regression introduces an 𝐿2
penalty that shrinks coefficients toward zero while retaining all regressors, thereby stabilizing
estimation in the presence of strong correlations. Lasso regression relies on an penalty, which𝐿1
can set some coefficients exactly to zero and thus performs variable selection in addition to
regularization. \\

All models are evaluated based on their predictive performance using out-of-sample criteria. The
primary loss functions considered are the Root Mean Squared Error (RMSE) and the Mean
Absolute Error (MAE), in accordance with the project guidelines. Comparing OLS with Ridge and
Lasso allows assessing the trade-off between interpretability and predictive accuracy in explaining
energy consumption from meteorological conditions.

\subsubsection{Estimation method}

In the cross-sectional framework devoted to energy consumption forecasting, several estimation
methods are considered in order to compare their predictive performance and to account for the
statistical properties of the meteorological explanatory variables.
The benchmark estimation method is Ordinary Least Squares (OLS). Under the standard
assumption of conditional exogeneity,
\begin{center}
    $\textbf{E}(u_i \mid X_i) = 0$,
\end{center}
the OLS estimator is unbiased and consistent. It provides a natural reference model, as it allows a
direct economic interpretation of the estimated coefficients and makes it possible to assess the
relevance of nonlinear effects, such as the quadratic term in temperature and the interaction
between temperature and humidity. In particular, OLS identifies average marginal effects of
meteorological variables on energy consumption.
However, several explanatory variables are potentially highly correlated, notably temperature, its
squared term, sunshine duration, and global radiation. This multicollinearity may inflate the
variance of OLS estimators and weaken out-of-sample predictive performance. To address this
issue, penalized regression methods are also implemented. \\
 \\
\textbf{Ridge regression} introduces an $L_2$ penalty on the coefficients and is defined as :
\begin{center}
    $\hat{\beta}^R = \textit{argmin}_{\beta} \sum_{i}^{}(Y_i - X_i\beta)^2 + \lambda\sum_{j \geq 1}^{}\beta_j^2$.
\end{center}
This penalization reduces the variance of the estimators in the presence of multicollinearity, at the
cost of a controlled bias, and improves the stability of the estimated coefficients. \\
 \\
\textbf{Lasso regression} relies on an $L_1$ penalty and is defined as:\par
\begin{center}
    $\hat{\beta}^L = \textit{argmin}_{\beta} \sum_{i}^{}(Y_i - X_i\beta)^2 + \lambda\sum_{j \geq 1}^{}\mid \beta_j \mid$.
\end{center}
Unlike Ridge regression, Lasso performs automatic variable selection by allowing some
coefficients to be exactly zero. This feature is particularly useful when not all meteorological
variables are equally relevant for predicting energy consumption.

Finally, model selection is based on out-of-sample predictive performance, in accordance with the
project guidelines. The OLS, Ridge, and Lasso models are compared using standard loss criteria
such as the Root Mean Squared Error (RMSE) and the Mean Absolute Error (MAE). The final
choice relies on a bias–variance trade-off and on the ability of the model to generalize beyond the
estimation sample.

% ------------------------ Estimation results ------------------------
\subsection{Estimation results}

This subsection presents the estimation results for daily energy consumption using a cross-sectional
framework. The dependent variable is total daily energy consumption, measured in megawatts,
constructed as the sum of gross electricity and gas consumption. It provides a comprehensive
indicator of overall energy demand.

The explanatory variables are exclusively meteorological and include average temperature, average
humidity, precipitation, wind speed, global solar radiation, and sunshine duration. These variables are
selected based on their expected influence on heating, cooling, lighting, and other weather-sensitive
energy uses. This choice is fully consistent with the exploratory analysis, which highlights strong
correlations between weather conditions and energy consumption.

Given evidence of non-linearity in the temperature–energy relationship, the empirical specification
incorporates a quadratic temperature term as well as an interaction between temperature and
humidity. These terms allow the model to capture increased energy demand at temperature extremes
and to account for the fact that temperature effects may vary with humidity levels. All continuous
variables involved in non-linear transformations are centered to improve numerical stability and
facilitate interpretation in the presence of multicollinearity.

The baseline model is estimated using ordinary least squares (OLS). In addition, Ridge and Lasso
regressions are implemented to assess robustness to multicollinearity and to compare out-of-sample
predictive performance.

The OLS estimation results show that meteorological variables explain a substantial share of the
variation in energy consumption. The adjusted R² is high, and the global F-test strongly rejects the null
hypothesis of no joint explanatory power. Temperature emerges as the dominant determinant of energy
demand: the linear temperature coefficient is negative, while the quadratic term is positive and highly
significant, revealing a clear U-shaped relationship. Energy consumption increases at low
temperatures due to heating needs and at high temperatures due to cooling demand, a result that
aligns closely with economic intuition.

Humidity also plays a significant role. The interaction between temperature and humidity is statistically
significant, indicating that high humidity amplifies the effect of temperature on energy demand,
particularly during hot periods. Variables related to solar exposure exhibit contrasting effects: sunshine
duration is positively associated with energy consumption, while global solar radiation has a negative
coefficient, reflecting potentially different channels such as lighting needs or indirect temperature
effects. By contrast, precipitation and wind speed do not appear statistically significant once other
meteorological variables are controlled for.

Diagnostic analysis reveals substantial multicollinearity among regressors, especially those related to
temperature and solar exposure, as indicated by a high condition number. While this does not
invalidate the model from a predictive standpoint, it motivates the use of penalized regressions. Ridge
and Lasso models are therefore estimated and evaluated on a test sample using RMSE and MAE. The
results show that all three models deliver very similar out-of-sample performance, with no meaningful
gain from regularization. Consequently, the OLS specification is retained as the preferred forecasting
model, as it combines strong predictive performance with simplicity and economic interpretability.

It is important to emphasize that these results are interpreted strictly in a predictive sense. The
estimated coefficients capture statistical associations useful for forecasting and do not represent
causal effects. Overall, the analysis confirms that meteorological variables—especially temperature
and humidity—are key drivers of energy consumption and that relatively parsimonious models can
yield reliable forecasts when properly specified and evaluated out of sample.

% ------------------------ Comparison of Models and Choice of the Forecasting Method ------------------------
\subsection{Comparison of Models and Choice of the Forecasting Method}

Several models were estimated to predict total energy consumption using meteorological variables.
The approaches considered include a linear regression estimated by Ordinary Least Squares (OLS), as
well as two penalized regression models, Ridge and Lasso.

Model selection is based exclusively on out-of-sample predictive performance, assessed using
forecasting loss functions adapted to prediction problems, namely the Root Mean Squared Error
(RMSE) and the Mean Absolute Error (MAE), in accordance with the project guidelines.

The results indicate that the predictive performances of the three models are overall very close. The
OLS model provides competitive forecasts, while the Ridge and Lasso models do not lead to a
substantial improvement in terms of RMSE or MAE on the test sample.

In this context, the OLS model is retained as the reference model, as it offers a good compromise
between predictive accuracy, simplicity, and economic interpretability. The penalized models
nevertheless remain informative : Ridge regression helps stabilize coefficient estimates in the
presence of multicollinearity, while Lasso highlights that some meteorological variables have a limited
marginal contribution to prediction.

Consistent with the objective of the project, this choice is driven by predictive performance rather than
by strict causal interpretation of the estimated coefficients. \\
 \\
\textbf{Methodological Remark}

The final model selection relies exclusively on out-of-sample predictive performance criteria, rather than on
in-sample goodness-of-fit measures or the individual statistical significance of estimated parameters.
This methodological choice is fully consistent with the project instructions and with the spirit of the course,
which emphasize a rigorous evaluation of forecasting performance over causal interpretation of model
coefficients.

% ------------------------ Final forecasts and comparative analysis ------------------------
\subsection{Final forecasts and comparative analysis}

Final forecasts of energy consumption are produced using the model retained at the end of the
selection phase. Predictions are generated on the test sample using observed meteorological variables
and constitute the final predictive output of the cross-sectional approach.

The predicted trajectory of energy consumption is globally consistent with observed levels in the out-
of-sample period. Forecasts correctly reproduce the order of magnitude of consumption as well as its
main variations, while exhibiting greater dispersion during episodes of high demand. This behavior
suggests that, although meteorological conditions contain substantial predictive information, certain
extreme demand episodes remain difficult to anticipate using these variables alone.
\newpage
\begin{figure}[h]
\centering
\includegraphics[width=1\textwidth]{../results/figures/Paligwende_selected_prediction_energy.png}
\caption{Forecast of energy consumption: observed Vs predicted values (test sample)}
\label{fig:RealVsPredicted}
\end{figure}

This figure illustrates the relationship between observed and predicted energy consumption values on
the test sample. Most observations lie close to the line of perfect prediction, indicating overall
coherence between forecasts and realizations. The largest deviations appear at the highest
consumption levels, highlighting model limitations during periods of exceptional demand.
Overall, these forecasts show that the retained model provides a plausible representation of out-of-
sample energy consumption and can be used to anticipate demand levels based on observed
meteorological conditions, within a strictly predictive framework. \\
\\
\textbf{Comparative Analysis}

The energy consumption forecasting exercise illustrates the strengths and limitations of a cross-
sectional approach based on exogenous meteorological information. By incorporating
contemporaneous weather variables, the model provides a coherent and economically interpretable
anticipation of average demand levels. This framework performs well under typical conditions,
capturing the main drivers of energy consumption related to temperature and humidity. However, its
predictive accuracy weakens during episodes of exceptional consumption, which may be driven by
extreme weather events, behavioral responses, or structural factors not fully captured by
meteorological variables. These results underline the limits of static cross-sectional models when
applied to rare or atypical situations.

% ------------------------ Robustness analysis and model limitations ------------------------
\subsection{Robustness analysis and model limitations}

\textbf{Robustness of the results}

The robustness of energy consumption forecasts is assessed by estimating several alternative
specifications, including a linear OLS model, non-linear extensions incorporating quadratic and
interaction terms, and regularized models such as Ridge and Lasso. Across these specifications, out-
of-sample predictive performance remains broadly similar, as measured by RMSE and MAE. This
stability indicates that the forecasts are not overly sensitive to the specific estimation method,
provided that the model structure is consistent with the underlying data-generating process. In
particular, the inclusion of non-linear temperature effects appears more important for predictive
accuracy than the choice between OLS and penalized estimators. \\
 \\
\textbf{Model and data limitations}

Despite these encouraging results, several limitations must be acknowledged. First, energy
consumption is explained exclusively by meteorological variables, whereas demand is also influenced
by economic activity, institutional settings, prices, and behavioral factors that are not observed in the
dataset. This omission may reduce predictive accuracy during episodes of exceptional consumption.
Second, the cross-sectional framework does not explicitly model the temporal dynamics of energy
demand, thereby excluding potential intertemporal dependence that could improve forecasts in some
contexts. Finally, results depend on data quality and aggregation choices, which may mask local
heterogeneity or short-term shocks. \\
 \\
\textbf{Implications for interpretation}

As a consequence, energy consumption forecasts should be interpreted as indicative predictions
capturing average demand responses to weather conditions rather than exact forecasts. They are best
suited for anticipating general demand levels and relative variations, particularly under typical
meteorological conditions.

% ------------------------------------------------ Conclusion ------------------------------------------------
\section{Conclusion}

This project aimed to explore and compare predictive methods applied to two distinct settings: forecasting
energy consumption based on meteorological variables and forecasting temperature using time series models.
By combining a cross-sectional and a temporal approach, the analysis highlights the importance of aligning
data structure, methodological choices, and predictive objectives.

The results show that meteorological variables provide relevant information for anticipating energy
consumption. A relatively simple regression model enriched with non-linearities generates coherent out-of-
sample forecasts and captures the main variations in energy demand. However, some extreme situations
remain difficult to predict, underscoring the limitations inherent in models relying solely on observable
climatic factors.

In parallel, temperature analysis reveals a strongly structured temporal dynamics dominated by stable annual
seasonality. The retained time series models effectively reproduce this structure and provide plausible short-
and medium-term forecasts. Explicit treatment of uncertainty through prediction intervals nonetheless
highlights the declining precision of forecasts as the horizon increases.

Beyond specific results, this project emphasizes the importance of a rigorous approach based on out-of-sample
evaluation, a clear separation between estimation, model selection, and forecasting, and explicit
acknowledgment of model limitations. It also shows that relatively simple models, when properly specified
and used within a predictive framework, can yield informative and operational results.

Finally, the project highlights the complementary nature of cross-sectional and time series approaches. Their
joint use allows forecasting strategies to be adapted to the nature of the phenomenon under study, while
reminding that all forecasts must be interpreted with caution and within the assumptions underlying the models
employed.


% ------------------------ References ------------------------
\begin{thebibliography}{9}

\bibitem{Meteo_france}
Météo France. \href{https://www.data.gouv.fr/datasets/donnees-climatologiques-de-base-quotidiennes}
{\textit{Données climatologiques de base - quotidiennes}} Licence Ouverte / Open Licence version 2.0.

\bibitem{feat_desc}
\begin{itemize}
    \item \textbf{RR}: Quantité de précipitation tombée en 24 heures, mesurée en millimètres et 1/10.
    \item \textbf{TM}: Moyenne quotidienne des températures horaires, exprimée en degrés Celsius avec une précision de 1/10.
    \item \textbf{FFM}: Moyenne quotidienne de la force du vent, mesurée en mètres par seconde (m/s) et moyenne sur 10 minutes, à 10 m de hauteur.
    \item \textbf{INST}: Durée d'insolation quotidienne mesurée en minutes.
    \item \textbf{GLOT}: Rayonnement global quotidien, exprimé en joules par centimètre carré (J/cm$^2$).
    \item \textbf{UM}: Moyenne quotidienne des humidités relatives horaires, exprimée en pourcentage (\%).
\end{itemize}

\bibitem{ODRE}
ODRE. \href{https://www.data.gouv.fr/datasets/consommation-quotidienne-brute-regionale}
{\textit{Consommation quotidienne brute régionale}} Licence Ouverte / Open Licence version 2.0.

\bibitem{corr_matrix}
\begin{figure}[h]
\centering
\includegraphics[width=1\textwidth]{../results/figures/Noa_correlation.png}
\caption{Correlation Matrix}
\label{fig:correlation}
\end{figure}

\bibitem{SDES}
SDES. \href{https://www.statistiques.developpement-durable.gouv.fr/edition-numerique/chiffres-cles-energie/fr/6-bilan-energetique-de-la-france}
{\textit{Bilan énergétique de la France | Chiffres clés de l'énergie - Édition 2025.}}

\bibitem{Whatever}
Auffhammer, M., Baylis, P., \& Hausman, C. (2017). \textit{Climate change is projected to have severe impacts on
the frequency and intensity of peak electricity demand across the United States. Proceedings of the National
Academy of Sciences}.

\bibitem{Whatever 2}
Bessec, M., \& Fouquau, J. (2008). \textit{The non-linear link between electricity consumption and temperature in
Europe}. Energy Economics.

\bibitem{Whatever 3}
Box, G. E. P., Jenkins, G. M., Reinsel, G. C., \& Ljung, G. M. (2015). Time Series Analysis: Forecasting and
Control. Wiley.

\bibitem{Whatever 4}
Deschênes, O., \& Greenstone, M. (2011). \textit{Climate change, mortality, and adaptation: Evidence from annual
fluctuations in weather in the US}. American Economic Journal: Applied Economics.

\bibitem{Whatever 5}
Hong, T., Pinson, P., \& Fan, S. (2016). \textit{Global energy forecasting competition 2012. International Journal of
Forecasting}.
\end{thebibliography}

\end{document}